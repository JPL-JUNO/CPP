\chapter{}
\section{Starting with Main}
When a computer runs code, it does so in a step-by-step, line-by-line manner. But your code is organized into pieces, and one of these pieces is the main function, or simply main(), which is the part that runs first. main() tells the computer
which other parts of the application you want to use. main() is the head honcho, the big boss.

How does the computer know what is main()? You type lines of code between the brace characters, \{ and \}.

\section{Showing Information}
\subsection{Tabbing your output}
Really want to display a backslash, not a special character? Use a backslash followed by another backslash. (Yes, it's bizarre.) The compiler treats only the first backslash as special. When a string has two backslashes in a row, the compiler treats the second backslash as, well, a backslash.